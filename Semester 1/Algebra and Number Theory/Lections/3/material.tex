
\DeclareRobustCommand{\divby}{%
  \mathrel{\vbox{\baselineskip.65ex\lineskiplimit0pt\hbox{.}\hbox{.}\hbox{.}}}%
}

\lesson{3}{22.09.2023}{Группы, кольца, поля и теория чисел}

\section{Группы}

\begin{eg}
    \begin{enumerate}
        \item $\R^{*} = (\R \setminus \{0\} , \cdot  )$ - абелева группа
        

        аналогично с $\Q^*, \Q^{*}_{+}, \R^{*}_{+}$
        \item $(\R, +)$ - абелева
        \item пусть X - множество, G - множество биекций $X \Rightarrow X, \circ$ - композиция, тогда G - группа
        \item Группа движений плоскости, операция - $\circ$
        \item пусть X - множество, тогда $(2^{X}, \triangle)$ - группа (доказать)
    \end{enumerate}
\end{eg}

\begin{property} (сокращение), G - группа, $a, b, c \in G$
    \begin{enumerate}
        \item если $ac = bc \Rightarrow a = b$
        \item если $ca = cb \Rightarrow a = b$
    \end{enumerate}
    \begin{proof}
        $ac = bc \overset{\exists c^{-1}}{\Rightarrow} (ac)c^{-1} = (bc)c^{-1} \overset{\text{ассоц.}}{\Rightarrow} a(cc^{-1}) = $
        
        $b(cc^{-1}) \Rightarrow ae = be \Rightarrow a = b$ Q.E.D.
    \end{proof}
\end{property}

\begin{definition}
    Группы G и H - $\underline{\text{изоморфны}}$, если $\exists$ биекция из G в H, т.ч. $\forall x, y \in G: f(x \cdot  y) = f(x) * f(y)$
    где $\cdot$ - операция G, * - операция H
\end{definition}

\begin{notation}
    $G \cong H$, f - изоморфизм
\end{notation}

\begin{eg}
    G$(\R, +)$
    H$(\R^{*}_{+}, \cdot)$
    $f(x) = 2^x$ - изоморфизм: 
    \begin{align*}
        f(x+y) &= 2^{x+y} \\
        f(x)f(y) &= 2^x \cdot 2^y
    \end{align*}
\end{eg}


\section{Кольца и поля}


\begin{remark}
    в теории чисел все числа по умолчанию целые
\end{remark}

\begin{definition}
    число a делится на b, если:
    $\exists c: a= bc$
\end{definition}

\begin{property}
    \begin{enumerate}
        \item $a \divby c, b \divby c \Rightarrow a + b \divby c, a - b \divby c$
        \begin{proof}
            $ a \divby c \Rightarrow a = kc \land b \divby c \Rightarrow b = mc$
            
            $a = kc \land b = mc \Rightarrow \begin{cases}
                a + b = (m + k)c \divby c \\
                a - b = (m - k)c \divby c
                \end{cases}$ Q.E.D.
        \end{proof}
        \item $ \forall k: a \divby b \Rightarrow ak \divby b$
        \item $a \divby b \land b \divby c \Rightarrow a \divby c$
        \item $a \divby b \Rightarrow |a| \geq |b| \lor a = 0$
        \begin{proof}
            $a = bc \Rightarrow$
            $\left[ 
                \begin{gathered} 
                c = 0, \text{значит } a = 0\\ 
                c \neq 0, \text{значит } |c| \geq 1\\ 
                \end{gathered} 
            \right.$
            
            $\text{значит, } |a| = |c||b| \geq |b|$ Q.E.D.
        \end{proof}
        \item $\forall a: a \divby 1$
        \item $\forall a: 0 \divby a$
    \end{enumerate}
\end{property}

\begin{definition}
    НОД $(a_1, a_2, \ldots, a_k)$ - наибольшее число, на которое делятся $a_1, a_2, \ldots, a_k$

    Обозначается как: $(a_1, a_2, \ldots, a_k)$
\end{definition}

\begin{definition}
    НОК $(a_1, a_2, \ldots, a_k)$ - наименьшее число, которое делится на $a_1, a_2, \ldots, a_k$

    Обозначается как: $[a_1, a_2, \ldots, a_k]$
\end{definition}

\begin{theorem}
    Если не все числа $a_1, a_2, \ldots, a_k$ равны нулю, но НОД существует.
\end{theorem}

\begin{proof}
    Пусть А - множество всех общих делителей, тогда $1 \in A \Rightarrow A \neq \varnothing$

    А ограничено сверху, т.к. $\forall \text{делитель } \leq |a_i|$, где $a_i$ - любое ненулевое число, значит, в множестве А есть наибольший элемент Q.E.D.
\end{proof}

\begin{theorem}
    Если все числа $a_1, a_2, \ldots, a_k$ не равны нулю, но НОК существует.
\end{theorem}



\begin{proof}
    Пусть А - множество всех общих кратных, тогда $a_1, a_2, \ldots, a_k \in A \Rightarrow A \neq \varnothing$

    А ограничено снизу числом 0, значит, в множестве А есть наименьший элемент Q.E.D.
\end{proof}

\section{Алгоритм Евклида}

\begin{theorem} (деление с остатком)
    Пусть $b \in \N, a \in \Z$, тогда $\exists! q, r$: 
    $\begin{cases} 
        a= bq + r,\\ 
        0 \leq r \leq b - 1
    \end{cases}$
\end{theorem}

\begin{proof}
    \begin{enumerate}
        \item Пусть $A = \{a - bx : x \in \Z\}$
        
            Среди элементов А есть хотя бы один неотрицательный:
            \begin{enumerate}
                \item[.] если $a \geq 0$, то $a \in A$
                \item[.] если $a <  0$, то $a - ab = a(1 - b) \in A$
            \end{enumerate}

            Пусть r - наименьший неотрицательный элемент в А. Проверим, что он подходит.

            $r = a - bx \Rightarrow a = bx + r$, х можно взять в качестве q 

            Преположим, что $r \geq b$, тогда: 

            $r - b = a - b(x + 1) \in A \Rightarrow$ r - не наименьший элемент в А $\Rightarrow r \leq b - 1$ 
        \item Докажем единственностью Пусть $a = bq_1 + r_1 = bq_2 + r_2;$
        
            $0 \leq r_1, r_2 \leq b - 1$

            $b(q_1 - q_2) = r_2 - r_1 \Rightarrow (r_2 - r_1) \divby b \Rightarrow \left[ 
                \begin{gathered} 
                    r_2 - r_1 = 0\\ 
                    |r_2 - r_1| \geq b - \text{противоречие: } r_1; r_2 \leq b - 1\\ 
                \end{gathered} 
            \right.$
            
            Значит, $r_1 = r_2 \Rightarrow q_1 = q_2$  Q.E.D.
    \end{enumerate}
\end{proof}

\begin{definition} (Алгоритм Евклида)
    даны числа $a, b \in \N, a \geq b$
    \begin{enumerate}
        \item если $a \divby b$ - конец алгоритма, результат = b 
        \item если же не делится, то алгоритм применяется к паре (b, r), где r - остаток от деления a на b
    \end{enumerate}
\end{definition}

\begin{eg}
    a = 22, b = 6
    \begin{enumerate}
        \item $22 = 3 \cdot  6 + 4: (22, 6) \to (6, 4)$
        \item $6 = 1 \cdot 4 + 2: (22, 6) \to (4, 2)$
        \item $4 = 2 \cdot 2$ - конец, ответ: 2
        
    \end{enumerate}
\end{eg}

\begin{remark} (Запись с формулами:)
    
    
    $a = bq_0 + r_1  \;\;\;\;\;\;\;\;\;\;\;\;\;\;    0 \leq r_1 < b$
    
    $b = r_1q_1 + r_2  \;\;\;\;\;\;\;\;\;\;\;\;\;    0 \leq r_2 < r_1$
    
    $r_1 = r_2q_2 + r_3  \;\;\;\;\;\;\;\;\;\;\;    0 \leq r_3 < r_2$
    
    
    $\vdots \;\;\;\;\;\;\;\;\;\;\;\; \vdots \;\;\;\;\;\;\;\;\;\;\;\;\;\;\;\;\;\;\;\;\;\;\; \vdots$

    $r_{k-2} = r_{k-1}q_{k-1} + r_k  \;\;\;\;  0 \leq r_k <r_{k-1}$

    $\vdots \;\;\;\;\;\;\;\;\;\;\;\; \vdots \;\;\;\;\;\;\;\;\;\;\;\;\;\;\;\;\;\;\;\;\;\;\; \vdots$

    $r_{n-2} = r_{n-1}q_{n-1} + r_n   \;\;\;\;  0 \leq r_n <r_{n-1}$

    $r_{n-1} = r_nq_n$, ответ: $r_n$
\end{remark}