
%\setcounter{chapter}{-1}

\chapter{Построение множества вещественных чисел}

\lesson{1}{14.09.2023}{Введение}

\section{Множества}

\begin{definition}
    Множества X и У равны, если:
    
    $\forall a \in X: a \in Y$

    $\forall b \in Y: b \in X$
\end{definition}

\begin{definition}
    $X \subset Y$ если:
    
    $\forall a \in X: a \in Y$
\end{definition}

\begin{definition}
    \begin{enumerate}
        \item $a \in A \cup B \Leftrightarrow a \in A \lor a \in B$
        \item $a \in A \cap B \Leftrightarrow a \in A \land a \in B$
        \item $a \in A \setminus B \Leftrightarrow a \in A \land a \notin B$
    \end{enumerate}
\end{definition}

\begin{definition} (Декартово произведение множеств)
    
    $A \times B = \{(a, b): \forall a \in A, \forall \in B\}; A, B \neq \varnothing$
\end{definition}

\begin{definition}

    $F: A \rightarrow B$ - функция, такая, что: $\forall a \in A$ сопостовляет $b = F(a) \in B$
\end{definition}

\section{Сечения}

\begin{definition}
    
    Множество $\alpha \subset \Q$ называется сечением, если:
    \begin{itemize}
        \item I. $\alpha \neq \varnothing$
        \item II. если $p \in \alpha, $ то $q < p \Leftrightarrow q \in \alpha$
        \item III. в $\alpha$ нет наибольшего
    \end{itemize}
\end{definition}

\begin{eg}
    \begin{enumerate}
        \item $p^* = \{r \in \Q: r < p\}$ - нет наибольшего
        \item $\sqrt{2} = \{p \in \Q: p \leq 0 \lor p > 0 \land p^2 < 2\}$ 
    \end{enumerate}
\end{eg}

\begin{theorem} (Утверждение 1)
    \label{Statement1}
    Если $p \in \alpha \land q \notin \alpha$, то $ q > p$
\end{theorem}

\begin{proof}
    Если $p \in \alpha$ и $q \leq p$, то из (II.) следует. что $q \in \alpha$
\end{proof}

\begin{theorem} (Утверждение 2)
    $\alpha < \beta \land \beta < \gamma \Rightarrow \alpha < \gamma$
\end{theorem}

\begin{proof}
    $\begin{cases}
        \alpha < \beta \Rightarrow \exists p \in \beta, p \notin \alpha \\
        \beta < \gamma \Rightarrow \exists p \in \gamma, q \notin \beta
    \end{cases}$
    $\Rightarrow p < q \Rightarrow \alpha < \gamma$
\end{proof}

\begin{theorem}
    Пусть $\alpha, \beta$ - сечения. Между ними существует одно из нескольких отношений:
    $\left[ 
        \begin{gathered} 
        \alpha < \beta \\ 
        \beta > \alpha \\
        \alpha = \beta 
        \end{gathered} 
    \right.$
\end{theorem}

\begin{proof}
    Предположим, что $\alpha < \beta$ и $\beta < \alpha$, тогда:

    $\begin{cases}
        \exists p \in \alpha, p \notin \beta \\
        \exists q \in \beta, q \notin \alpha  
    \end{cases}$
    $\Rightarrow$
    $\begin{cases}
        p > q \\
        q > p
    \end{cases}$ - Противоречие, тогда $\alpha \neq \beta$
\end{proof}

\section{Сумма сечений}

\begin{theorem}
    Пусть $\alpha, \beta$ - сечения, тогда:

    $\alpha + \beta = \{p + q: p \in \alpha, q \in \beta\}$ - тоже сечение.
\end{theorem}

\begin{proof}
    
    \begin{itemize}
        
        
        \item (I.) Пусть $\exists s \notin \alpha, \exists t \notin \beta$, тогда:

        $\forall p \in \alpha, q \in \beta:$
        $\begin{cases}
            p < s \\
            q < t
        \end{cases} \Rightarrow p + q < s + t \Rightarrow \alpha + \beta \neq \Q$
        \item (II.) 
        
        $r \in \alpha + \beta, r_1 < r$
        
        $r = p + q, p \in \alpha, q \in \beta$

        $r_1 = p + q_1, r_1 < r \Rightarrow q_1 < q \Rightarrow q_1 \in \beta \Rightarrow p + q_1 \in \alpha + \beta$
        \item (III.)
        
        $\exists p_1 \in \alpha, p > p_1 \Rightarrow p_1 + q > p + q = r, p_1 + q \in \alpha + \beta$ - нет наибольшего
    \end{itemize}
\end{proof}

\begin{theorem} (Свойства суммы сечений)
    \begin{enumerate}
        \item $\alpha + \beta = \beta + \alpha$
        \item $(\alpha + \beta) + \gamma = \alpha + (\beta + \beta)$
        \item $\alpha + 0^* = \alpha$, где $0^* = \{p \in \Q: p < 0\}$
    \end{enumerate}
\end{theorem}

\begin{proof}
    Свойства 1 и 2 справедливы в силу коммутативности и ассоциативности рациональных чисел.
    
    Докажем свойство 3:
    \begin{enumerate}
        \item Пусть $p \in \alpha, q \in 0^* $, тогда: $ p + q < p \Rightarrow p + q \in \alpha$, т.е. $\alpha + 0^* \subset \alpha$
        \item Пусть $p \in \alpha$, тогда: $\exists p_1 > p \Rightarrow p_1 \in \alpha, p = p_1 + (p - p_1)$, при том $p_1 \in \alpha, p - p_1 \in 0^* \Rightarrow p \in \alpha + 0^* \Rightarrow \alpha \subset \alpha + 0^*$
    \end{enumerate}

    $\begin{cases}
        \alpha \subset \alpha + 0^* \\
        \alpha + 0^* \subset \alpha
    \end{cases} \Rightarrow \alpha = \alpha + 0^*$
\end{proof}

\section{Теоремы сечений}

\begin{theorem} (Теорема 2)
    Пусть $\alpha$ - сечение, $r \in \Q^+$, тогда $\exists p \in \alpha \land q \notin \alpha$:

    q - не наименьшее верхнее (не входящее в сечение) число

    $q - p = r$
\end{theorem}

\begin{proof}
    Пусть $p_0 \in \alpha, p_1 = p_0 + r$

    \begin{enumerate}
        \item Возможно, $p_1 \notin \alpha$, тогда:
        
        \begin{enumerate}
            \item если $p_1$ - не наименьшее в верхнем классе, то $q = p_1$
            \item если же наименьшее, то $p = p_0 + \frac{r}{2}, q = p_1 + \frac{r}{2}$
        \end{enumerate}
        \item Если $p_1 \in \alpha$, тогда:
        
        Положим $p_n = p_1 + nr$ для $n = 0, 1, 2, \ldots$. Тогда $\exists! m$: 
        
        $p_m \in \alpha$ и $p_{m+1} \notin \alpha$

        \begin{enumerate}
            \item Если $p_{m+1}$ - не наименьшее в верхнем классе, то выберем $p = p_m, q = p_{m+1}$
            \item Если же наименьшее, то $p = p_m + \frac{r}{2}, q = p_{m+1} + \frac{r}{2}$
        \end{enumerate}
    \end{enumerate}
\end{proof}

\begin{theorem} (Существование противоположного элемента)
    Пусть $\alpha$ - сечение, тогда $\exists! \beta: \alpha + \beta = 0^*$
\end{theorem}

\begin{proof} (нужно доказать единственность и существование)
    \begin{enumerate}
        \item Докажем единственность: пусть $\exists \beta_1, \beta_2$, удовлетворяющие условию, тогда:
        
        $\beta_2 = 0^* + \beta_2 = (\alpha + \beta_1) + \beta_2 = (\alpha + \beta_2) + \beta_1 = 0^* + \beta_1 = \beta_1$
        \item Докажем существование: пусть 
        
        $\beta = \{p: -p \notin \alpha, -p \text{ не является наименьшим в верхнем классе }\alpha\}$
        \begin{itemize}
            \item (I.) Очевидно, что $\beta \neq \varnothing, \Q$
            \item (II.) Возьмем $p \in \beta, q < p \Leftrightarrow -q > -p \Rightarrow -q$ в верхнем классе $\alpha$, но не наименьшее $\Rightarrow q \in \beta$
            \item (III.) Если $p \in \beta$, то -p - не наименьшее в верхнем классе $\alpha$, значит $\exists q: -q < -p$ и $-q \notin \alpha$
            
            Положим $r = \frac{p+q}{2}$, тогда:

            $-q < -r < -p \Rightarrow$ -r -- не наименьшее в верхнем классе $\alpha$. Значит, нашли такое r > p, что $r \in \beta$
        \end{itemize}
    \end{enumerate}

    Теперь проверим, что $\alpha + \beta = 0^*$:

    \begin{enumerate}
        \item Возьмем $p \in \alpha, q \in \beta$

        По определению $\beta: -q \notin \alpha \underset{\text{Утв. 1}}{\Rightarrow} -q > p \Leftrightarrow p + q < 0 \Rightarrow p + q \in 0^* \Rightarrow \alpha + \beta \subset 0^*$ 
        \item Возьмем по Теореме (2) $q - p = r \Leftrightarrow p - q = -r \in 0^*$
        
        т.к. $q \notin \alpha$, то $-q \in \beta$, значит $p - q = p + (-q) \in \alpha + \beta \Rightarrow 0^* \subset \alpha + \beta$
    \end{enumerate}
    
    $\begin{cases}
        \alpha + \beta \subset  0^* \\
        0^* \subset \alpha + \beta
    \end{cases} \Rightarrow \alpha  + \beta = 0^*$
\end{proof}