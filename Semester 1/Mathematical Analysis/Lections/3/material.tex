
%\setcounter{chapter}{-1}


\chapter{Алгоритмы}

\lesson{3}{27.09.2023}{Продолжение}

\section{Продолжение}



5. $x_n \neq c \forall n, x_n \to a, a \neq 0 => \frac{1}{x_n} \to \frac{1}{a}$

$|a + b| \leq |a| + |b| <=> |a| \geq |a + b| - |b|$



$\epsilon_0 = \frac{|a|}{2} > 0$

$=> \exists N$ т.ч. $\forall n > N$ выполняется

$|x_n-a| < \epsilon_0 = \frac{|a|}{2} => |x_n| \geq |a| - |x_n - a| > |a| - \frac{|a|}{2} = \frac{|a|}{2}$

$\forall \epsilon \exists N_1$ т.ч. $\forall n > N_1$ (1)

$|x_n - a| < \epsilon$ (2)

$N_0 = max(N, N_1) n> N_0$

$|\frac{1}{x_n} - \frac{1}{a}| = |\frac{a - x_n}{x_n a} = \frac{1}{|a|} \cdot \frac{1}{|x_n|} \cdot |x_n - a| < $

(1, 2)

$< \frac{1}{|a|} \cdot \frac{2}{|a|} \cdot \epsilon$

6. ${x_n}_{n=1}^{\infty}$, как в 5., $y_n \to b => $

$\frac{y_n}{x_n} \to \frac{b}{a}$

$\frac{y_n}{x_n} = y_n \cdot \frac{1}{x_n}$
4., 5

7. $x_n \leq y_n \forall n, x||n \to a, y_n b => a \leq b$

\begin{proof}
    Предположим, что это не так.

    Пусть a $\ngtr$ (доказали что неверно) b (?)

    $\epsilon_0 = \frac{1 - b}{2} > 0$

    => $\exists N_1$ т.ч. $\forall n > N_1$


    $|x_n - a| < \epsilon_0$ (3)

    и $exists N_2$ т.ч $\forall n > N_2$

    $|y_n - b| < \epsilon_0$ (4)

    $n = N_1 + N_2 + 1$

    $|x_n - a| < \epsilon_0 <=> x_n \in (a - \epsilon_0, a + \epsilon_0)$ (3')

    $|y_n - b| < \epsilon_0 <=> y_n \in (b - \epsilon_0, b + \epsilon_0)$ (4')

    (3'), (4') => $y_n < b + \epsilon_0 = b + \frac{a - b}{2} = \frac{a + b}{2} = a \frac{a - b}{2}$

    $ = a - \epsilon_0 < x_n$

    $y_n < x_n$
\end{proof}


$a < b$

$(a, b) = \{x \in R: a < x < b\}$

$[a, b] = \{x \in R: a \leq x \leq b\}$

$[a, b) = \{x \in R: a \leq x < b\}$
$(a, b] = \{x \in R: a < x \leq b\}$

Расширенное множество вещественных чисел

$\overline \R$

$+ \infty, - \infty$

$\forall x \in \R$ $x < + \infty, x > - \infty$

(a, $\infty$) = $\{x \in \R : x > a\}$

[a, $\infty$) = $\{x \in \R : x \geq a\}$

(-$\infty$, a] = $\{x \in \R : x < a\}$

(-$\infty$, a] = $\{x \in \R : x \leq a\}$

8. $\xi_n \leq \psi_n \leq \zeta_n \forall n$

$\xi \to a, \zeta_n \to a => \psi_n \to a$

$\forall \epsilon > 0 \exists N_1$ т.ч. $\forall n > N_1$

$|x_n - 1| < \epsilon \leftrightarrow x_n \in (a - \epsilon, a + \epsilon)$ (5)

и $\exists N_2$ т.ч. $\forall n > N_2$

$|\zeta_n - a| < \epsilon \leftrightarrow \zeta_n \in (a - \epsilon, a + \epsilon)$ (6)

(5), (6) => $\forall n > N, N = max(N_1, N_2)$

$a - \epsilon < x_n \leq y_n \leq \zeta_n < a + \epsilon$, т.е. $ y_n \in (a - \epsilon, a + \epsilon) \leftrightarrow |y_n - a| < \epsilon$

\begin{definition}(Бесконечные пределы)

    $\{x_n\}_{n=1}^{\infty}$

    $x_n \to \infty$
    $n \to \infty$

    $\lim_{\to \infty} x_n = + \infty$

    если $\forall L \in \R \exists N$ т.ч. $\forall n > N$

    выполнено
    $x_n >L$(7)

    $\{y_n\}_{n=1}^\infty$

    $y_n \to -\infty$
    $n \to \infty$

    $\lim_{n \to \infty} y_n = -\infty$,

    $\forall L_0 \in R$, $\exists N_0$ т.ч. $\forall n > N_0$

    $y_n < L_0$ (8)

    (возможно сокращение записи n-> далее.)


\end{definition}

Единообразная запись определения пределов

$ a \in \R$

$w(a) = (a - \epsilon, a + \epsilon)$

Окрестность $+\infty$

$w(+\infty) = (L, \infty), L \in \R$

Окрестность $-\infty$

$w(-\infty) = (-\infty, L)$

Пусть имеется некая 
$\alpha \in \overline \R$

Пусть имеется некая последовательность
$\{x_n\}_{n=1}^\infty $

$x_n \to \alpha$
$n \to \infty$

если $\forall w (\alpha)$

$\exists N$ т.ч. $\forall n > N$ выполнено
$x_n \in 2 (\alpha)(q)$

Свойства бесконечных пределов

$\{a_n\}_{n=1}^\infty, a \to +\infty$

$\{b_n\}_{n=1}^\infty, b \to -\infty$

\begin{enumerate}
    \item $c \neq 0$, a) $c a_n \to + \infty, c b_n \to -\infty$
    

    б) $c < 0 => c a_n \to -\infty, c b_n \to +\infty$

    \item $x_n \to x$ , $x \in \R \cup \{+\infty\} => a_n + x_n \to +\infty$
    

    $y_n \to y$, $y \in \R \cup \{-\infty\} => b_n + y_n \to -\infty$


    \item $a_n, b_n, x_n, y_n, u_\epsilon2$
    
    $x > 0 => a_n x_n \to + \infty$, $b_n x_n \to -\infty$

    $y < 0 => a_n y_n \to -\infty, b_n y_n \to +\infty$

    \item если $a_n \neq 0, a_n \neq 0 \forall n => \frac{1}{a_n} \to 0, \frac{1}{b_n} \to 0$
    Если $x_n > 0, x_n \to 0 => \frac{1}{x_n} \to +\infty$

    если $y_n < 0, y_n \to 0 => \frac{1}{y_n} \to -\infty$


    \item $x_n \leq y_n \forall n, x \to \alpha, y_n \to \beta, \alpha, \beta \in \overline \R$
    
    => $\alpha \leq \beta$

    $+\infty = +\infty$

    $-\infty = -\infty$

    $-\infty < + \infty$

    $\alpha \in \overline \R$ => $y_n \to \alpha$

    (док-ть всё)

    \begin{proof}
        $x \in \R$ 

        если последоавтельность имеет предел, то она ограничена (было)

        нужно сформулировать с дополнительными словами

        Пусть $\{x_n\}_{n = 1}^\infty$ имеет конечный предел

        $\exists M$ т.ч. $|x_n - x| < M \forall n$

        => $x_n > x - M \forall n$ (10)

        $\forall L \in \R$

        $\exists N$ т.ч. $\forall n > N$ будет выполнено $a_n > L$ (11)

        (10), (11) => $a_n + x_n > L + x - M$

        Остальные свойства доказываются аналогично
    \end{proof}

\end{enumerate}


Дополнительно о терминологии и обозначениях

если $x_n \to 0$, то говорят что $x_n$ - бесконечно малая последовательность

если $|a_n| \to + \infty$, то говорят что $a_n$ - бесконечно большая последовательность



\begin{notation}

o - o малое

O - O Большое

\end{notation}

след. читать только слева направо.

\begin{notation}
    $x_n = o(1)$, если $x_n \to 0$

    если $\exists M > 0$ т.ч. $|y_n| \leq M \forall n$,

    $y_n = O(1)$


\end{notation}

$\{a_n\}_{n=1}^\infty $, $\{b_n\}_{n=1}^\infty $, $b_n \neq 0 \forall n$

$a_n = 0(b_n)$, если $\frac{a_n}{b_n} \to 0$

$\{c_n\}, \{d_n\}$

$c_n = O(d_n)$, если $\exists M_1$ т.ч.
$|C_n| \leq M_1 |d_n|$

предположим =, что
$a_n = \lambda_n b_n, \lambda_n \to 0$

Тогда пишут, что $a = o(b)n$

$\frac{a_n}{b_n} = \lambda_n$

\begin{definition}(монотонные последовательности)
    $\{a_n\}_{n=1}^\infty $ монотонно возрастает, если $a_n \leq a_{n+1} \forall n$

    Будем говорить, что строго возрастает, если $a_n < a_{n+1}$

    $\{b_n\}_{n=1}^\infty $ монотонно убывает, если $b_n \geq b_{n+1}$

    $\{b_n\}_{n=1}^\infty $ строго монотонно убывает, если $b_n > b_{n+1}$

    $\{с_n\}_{n=1}^\infty $

    Если есть некоторая поледовательнотсть $c_n$ говорят что монотонна если либо монотонно возрастает, либо монотонно убывает.

    Последовательность $c_n$ называется строго монотонной, если она строго монотонно возрастает либо строго монотонно убывает.

\end{definition}


\begin{theorem} Теорема о пределе монотонной последовательности
$\{C_n\}_{n=1}^\infty$

$\exists \lim_{n \to \infty} c_n \in \overline \R$

Для того чтобы монотонно возрастающая последовательность имела конечный предел необходимо и достаточно чтобы последовательность была ограничена снизу

Для того чтобы монотонно убывающая последовательность имела конечный предел.

$C_m \leq \lim_{n \to \infty} C_n \forall m$

$C_m < \lim_{n \to \infty} C_n$

$C_M \geq \lim_{n \to \infty} C_n$

$C_M  \lim_{n \to \infty} C_n$

\end{theorem}

\begin{proof}

Рассмотрим ситуация, когда $C_m$ монотонно возрастает.
Предположим вначалае, что проследовательность $C_m$ не ограничена сверху.

$\{C_n\}_{n=1}^\infty$ не огр. сверху
$\forall L \in \R$

Посколько мы предполгаем что последовательность не ограничена сверху значит найжется такой 
лемент послежовательности больший чем L

$\exists N$ т.ч. $C_N > L$

Потому что в противоположном случае L была бы верхней границей


$\forall n > N$ тогда, справедливо следующее неравенство $C_n \geq C_{n-1} \geq C_{n-2} \geq ... \geq C_N + 1 \geq C_N > L$  

т.е. $C_n >L$

мы взяли любое L и по нему нашли такое N большое, что при любом n > N полуается что c с номером n Больше чем lambda
это означает что по определению предела предел
$\lim C_n = +\infty$

Если последовательность возрастает и не ограничена сверху у нее есьт пределе и этот предел равен + бесконечности


другой вариант:
последовательность возрастает и огранчена сверху 

Пусть $C_n \leq C_{n+1} n \exists M т.ч. e_n \leq M \forall n$

рассмотрим множество всех элементов последовательности

$E = \{\alpha \in \R: \exists n \in \N$ т.ч. $\alpha = C_n$


Это предположение означает что E ограничено сверху

$ c = sup E $

в таком случае мы имеем неравенство
$C_n \leq C \forall n$ (12) 

Теперь возьмем $\forall \epsilon > 0$

$C - \epsilon$ - это не верхняя граница

$\exists N$ т.ч. $C_N > C - \epsilon$ (13)

Воспользуемся монотонностью последовательности C

Давайте возьмем $\forall n > N$

(13) => $C_n \geq C_{n-1} \geq ... \geq C_{N + 1} \ geq C_N > C- \epsilon$ (14)

Посмотрим на соотношение 12, 14 

$C - \epsilon < C_N \leq C < C + \epsilon => |C_n - C| < \epsilon$ (15)

Это соотношение означает что

(15) => $C = lim_{n \to \infty} C_n$

Предел существует, являющийся вещественным числом.

мы доказали что если последовательность ограничена сверху, то существует предел и выполенно такое неравенство.


\end{proof}


Если последовательность строго монотонна, то неравенство будет строгим


\begin{proof}
    $C_{n_0} < C_{n_0 + 1} \leq c => C_{n_0} < C$

\end{proof}

Если $\exists \lim_{n \to \infty} C_n = C \in R => \exists M$

т.ч. $|C_n - C| \leq M => C_n \leq C + M \forall n$

для убывающих доказывается аналогично.

\begin{theorem} (Теорема о ложных промежутках)
    $[a_n, b_n] \supset [a_{n + 1}, b_{n + 1}] \forall n$ (16)
\end{theorem}

Предположим, что 
$b_n - a_n \to 0$ (17)
$n -> \infty$

Промежутки замкнутые

=> $\exists! c \in [a_n, b_n], \forall n$ (18)

\begin{proof}
    $a_n \leq a_{n + 1}, b_n \geq b_{n + q} \forall n$ (19)

    $a_1 \leq a_2 \leq ... \leq a_n < b_n \leq b_{n-1} \leq ... \leq b_2 \leq b_1$ (19)

    $a_1 \leq a_n \leq b_n \leq b_1 \forall n$

    т.е. $a_n < b_1, b_n > a$, (20)

    (19), (20) => $\exists \lim_{n \to \infty} = a \in \R$
    и $\exists \lim_{n \to \infty} b_n = b \in \R$ (21)
    
    $a_n < b_n$

    => $lim_{n \to \infty} a_n \leq \lim_{n \to \infty} b_n$ (22)

    (21), (22) => $a \leq b$ (23)
    
    $a_n \leq a \forall n$
    $b_n \geq \forall n$

    (24)

    => $b - a \leq b_n - a_n \forall n$

    (25)

    $0 \leq $ b - a => $lim_{n \ to \infty} (b - a) \leq lim_{n \to \infty} (b_n - a_n) = 0$ (26)

    (23), (26) => a = b = def c

    (24), (27)=> $a_n \leq c \leq b_n \forall n, $ т.е. $c \in [a_n, b_n]$ (27')

    Пусть $\exists c_1 \neq c$ т.ч. $c_1 \in [a_n, b_n] \forall n$ (28)

    $c < c_1$

    Тогда, 27' и 28 => что $a_n \leq c < c_1 \leq b_n \forall n$ (29)

    (29) => $c_1 - c \leq b_n - a_n \forall n$ (30)

    (30) => $\lim_{n \to \infty} (c_1 - c) \leq \lim_{n \to \infty} (b_n - a_n) = 0$
    $0 < c_1 -c$ = 
    Предположение о том что найдется ещё какой-то $c_1$ неверно
    теорема доказана.

    





\end{proof}

\begin{remark}
    В этой теореме рассматриваются замкнутые Промежутки

    \begin{eg}
        $a_n = O \forall n, b_n = \frac{1}{n}$

        $(a_{n + 1}, b_{n + 1}) = (0, \frac{1}{n+1}) \subset (0, \frac{1}{n}) = (a_n, b_n)$

        $b_n - a_n = \frac{1}{n} \to 0$
        $n \to \infty$

        $\nexists C \in \R$ т.ч. $ c \in (0, \frac{1}{n}) \forall n$
    \end{eg}

    в каком месте доказательства предыдущей теоремы мы пользовались тем что промежутки замкнуты?
\end{remark}

\section{Число $e$}

$e$

$x_n = (1 + \frac{1}{n})^n$
$y_n = (1 + \frac{1}{n})^{n+1}$
$x_n < y_n \forall n $ (1)

$x_n$ строго возрастает (2)

$y_n$ строго убывает (3)

$x_n \to e, y_n \to e$

$2 < e < 3$

$y_n = (1 + \frac{1}{n}) x_n > x_n$

Рассмотрим $\frac{y_n - 1}{y_{n}} = \frac{(\frac{n}{n-1})^n}{(\frac{n+1}{n})^{n + 1}}$

$= (\frac{n}{n-1})^n \cdot (\frac{n}{n+ 1})^n + 1$

$\frac{n}{n + 1} \cdot (\frac{1}{n-1})^ n \cdot (\frac{1}{n+1})^ n$

$= \frac{n}{n + 1} \cdot (\frac{n^2}{n^2 - 1})^n$

$= \frac{n}{n+1} (\frac{n^2 - 1 + 1}{n^2 - 1})^n = \frac{n}{n+1} \cdot (1 + \frac{1}{n^2 - 1})^n >$

$(n^2 - 1 =\} x)$

$x> 0, n \geq 2$
$(1 + x)^n > 1 + nx$ ( неравенство бернулли)

$> \frac{n}{n + 1} (1 + \frac{n}{n^2 - 1})$

$= \frac{n}{n+1} \cdot \frac{n^2 - 1 + n}{n^2 - 1} = $

$= \frac{n^3 + n^2 - n}{n^3 + n^2 - n - 1} > 1$

$\frac{y_{n-1}}{y_n} > 1$

$y_{n-1} > y_n$

$(a + b)^n = \sum_{k = 0}^{n} C_n^k a^{n-k} b^k$

$C_n^k = \frac{n!}{k!(n-k!)}$

$C_n^0 = C_n^n = 1$

$C_n^1 = C_n^{n-1} = n$

$x_n = (1 + \frac{1}{n})^n = \sum_{k = 0}^{n} C_n^k (\frac{1}{n})^k = 1 \cdot 1 + n \cdot \frac{1}{n} + \sum_{k = 2}^{n} C_n^k \frac{1}{n^k}$

$= 2 + \sum_{k = 2}^{n} \frac{n!}{k!(n-k)!} \cdot \frac{1}{n^k} = 2 + \sum_{k = 1}^{n} \frac{1}{k!} \cdot \frac{(n=k+1) \cdot ... \cdot n }{n^k}$

$ = 2 + \sum_{k = 2}^{n} \frac{1}{k!} (1 - \frac{k - 1}{n})$

$= 2 + \sum_{n}^{k = 2} \frac{1}{k!} (1 - \frac{k - 1}{n})(1 - \frac{k - 2}{n})\cdot (1 - \frac{1}{n})$ (5)


$\frac{n - k + 1}{n} = 1 - \frac{k - 1}{n}$

$\frac{n - k + 2}{n} = 1 - \frac{k - 2}{n}$

...


$\frac{n - k + k}{n} = 1 - \frac{k - k}{n} = 1$

$\frac{n!}{(n-k)!}= \frac{(n-k)!(n-k+1) \cdot ... \cdot n}{(n - k)!} = (n - k + 1) \cdot ... \cdot n$




$n \geq 3$

$a = 1, b = \frac{1}{n}$

$1^{n - k} = 1$


